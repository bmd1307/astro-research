\documentclass[apj]{emulateapj}
%\usepackage{apjfonts}
\usepackage{graphicx}
\usepackage{amsmath,amssymb}
%\usepackage{Times}
%\usepackage{natbib}
\usepackage{amsmath}

\bibliographystyle{apj}

\def\mnras{MNRAS}
\def\apj{ApJ}
\def\aj{AJ}
\def\apjl{ApJL}
\def\aap{AAp}
\def\nat{Nature}
\def\pasj{PASJ}
\def\jcap{JCAP}
\def\prd{PRD}
\def\pasp{PASP}

\let\oldAA\AA
\renewcommand{\AA}{\text{\normalfont\oldAA}}

\begin{document}

\title{The Supernova Rate Beyond The Optical Radius}

\author{People (Brennan Dell, Sukanya Chakrabarti, Ben Lewis, Or Graur, Alex Filipenko)}

\begin{abstract}
Many spiral galaxies have extended outer HI disks, and display low levels of star formation, inferred from the FUV emission detected by GALEX, well beyond the optical radius. Here, we investigate the supernova rate (SN rate) in the outskirts of galaxies, using the largest and most homogeneous set of nearby supernova from the Lick Observatory Supernova Search (LOSS).    Up to now, supernova rates have been measured with respect to various galaxy properties, such as stellar mass and metallicity, but not with respect to galactocentric radius.  Understanding the SN rate as a function of intra-galactic environment has many ramifications -- from the interpretation of LIGO observations, the formation of massive stars, to the observed high velocity dispersion of the outer HI disk.  Using the data from the LOSS survey, we find that supernovae in dwarf galaxies (with a stellar mass less than $10^{9} M_{\odot}$) host 3.962 +/- 2.175 supernovae per millennium, while supernovae beyond the optical radius of spiral galaxies occur at a rate of 2.480 +/- 0.490 supernovae per millennium. The agreement in these figures suggests that the outskirts of spiral galaxies are just as likely to be hosts of binary black hole mergers as dwarf galaxies are. In the future, better measurements of the stellar mass in the outskirts of galaxies as well as larger homogeneous samples of supernova data should refine this measurement, and clarify the relationship between the supernova rate and galactocentric radius.
\end{abstract}

\section{Introduction}

The rate of supernovae is a fundamental quantity that has far-reaching implications for many areas of astrophysics and cosmology.  The SN rate is directly tied to the metal enrichment of galaxies, and the birth rate of neutron stars and black holes.  Earlier work has assembled a large and homogeneous set of nearby SNe (Leaman et al. 2011), i.e., the LOSS survey, that has been analyzed to search for correlations between the SN rate and galaxy properties (Graur et al. 2017).  Many spiral galaxies have extended HI disks (Walter et al. 2008), and display FUV emission in the outskirts (Thilker et al. 2007; Bigiel et al. 2010) that is indicative of a low level of star formation beyond the optical radius. Some of these stars will explode as supernovae, but the SN rate in the outskirts of galaxies has not yet been determined.  Here, we analyze the LOSS survey to directly determine the SN rate beyond the optical radius.

Recent LIGO observations (Abbott et al. 2016; Abbott et al. 2017) indicate that these merging massive binary black holes form in regions of low metallicity (Belczynski et al. 2016).  Several recent papers have now studied the host galaxies of binary black hole (BBH) mergers with binary merger population synthesis models (Lamberts et al. 2016; O'Shaughnessy et al. 2017; Chakrabarti et al. 2017).  Lamberts et al. (2016) showed that these events are either arising from dwarf galaxies or massive galaxies at high redshift.  O'Shaughnessy et al. (2017) noted that due to the low metallicity star formation found in dwarf galaxies, binary black hole mergers would be abundantly produced there.  Chakrabarti et al. (2017) found that the outskirts of spiral galaxies, which manifest a low level of star formation and have low metallicities, would contribute at least as much to the observed LIGO/Virgo detection rates as dwarf galaxies do.  If this is correct, one may expect that other tracers of the formation of massive stars should also be present in the outer disk, and the SN rate is one such tracer that we investigate here.

A puzzling observation of outer HI disks is that the velocity dispersion close to $R_{25}$ (where the B-band reaches 25-th magnitude) is about 10 km/s, which Tamburro et al. (2009) noted would require an unrealistic amount of supernova feedback efficiency if supernova were entirely responsible to maintaining the velocity dispersion in the outer disk.  Their estimate of the SN rate is however indirect, and is based on the observed star formation rate.  Here, we also investigate this puzzling observation, but with a direct calculation of the supernova rate beyond the optical radius.  

This paper is organized as follows.   In \S 2, we briefly review the parameters of the LOSS survey and summarize how we calculate SN rates here.  In \S 3, we present our main results, and we contrast in particular the SN rate in the outskirts of spiral galaxies with the rate in dwarf galaxies as a whole,  In \S 4, we discuss the ramifications of this result, and we conclude in \S 5.

\section{Methods}

I ran an interesting experiment

\section{Results}

\section{Discussion}

\section{Conclusion}

\end{document}
